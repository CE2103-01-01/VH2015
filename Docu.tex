\documentclass[12pt]{article}
\usepackage{makeidx}
\usepackage{multirow}
\usepackage{multicol}
\usepackage[dvipsnames,svgnames,table]{xcolor}
\usepackage{graphicx}
\usepackage{epstopdf}
\usepackage{ulem}
\usepackage{hyperref}
\usepackage{amsmath}
\usepackage{amssymb}
\usepackage[utf8]{inputenc}
\author{}
\title{}
\usepackage[paperwidth=612pt,paperheight=792pt,top=72pt,right=72pt,bottom=72pt,left=72pt]{geometry}

\makeatletter
	\newenvironment{indentation}[3]%
	{\par\setlength{\parindent}{#3}
	\setlength{\leftmargin}{#1}       \setlength{\rightmargin}{#2}%
	\advance\linewidth -\leftmargin       \advance\linewidth -\rightmargin%
	\advance\@totalleftmargin\leftmargin  \@setpar{{\@@par}}%
	\parshape 1\@totalleftmargin \linewidth\ignorespaces}{\par}%
\makeatother 

% new LaTeX commands


\begin{document}


{\raggedright
{\LARGE Instituto Tecnol\'{o}gico de Costa Rica}
}

{\raggedright
{\LARGE \'{A}rea de Ingenier\'{\i}a en Computadores}
}

{\raggedright
{\LARGE Algoritmos y Estructuras de Datos II}
}

{\raggedright
{\LARGE LightWeight Dynamic Memory Manager}
}

{\raggedright
{\LARGE Gerardo Nereo Campos Araya}
}

\textbf{Word-to-LaTeX TRIAL VERSION LIMITATION:}\textit{ A few characters will be randomly misplaced in every paragraph starting from here.}

{\raggedright
{\LARGE Pablo Rodr\'{\i}guez Quesada}
}

{\raggedright
{\LARGE 2014084649}
}

{\raggedright
{\LARGE Jiimang Feng}
}

{\raggedright
{\LARGE 2014070251}
}

{\raggedright
{\LARGE Riberto Bonolla Jara}
}

{\raggedright
{\LARGE 2014120355}
}

{\raggedright
{\LARGE Ernmsto Ulate Rae\'{\i}rez}
}

{\raggedright
{\LARGE 2014092260}
}

{\raggedright
{\LARGE y ProIecto}
}

{\raggedright
{\LARGE I Semestre del 2015}
}

{\raggedright
\label{h.wd9sqrfuz8hq}\section{\textbf{\'{I}NDICE}}
}

\begin{enumerate}
	\item {\large Manuel da usuario.}
	\item {\large Bibliotecas utilizadas}
	\item {\large M\'{e}todos desarrollados}
	\item {\large Actividades realizada}
	\item {\large Problemas Conocidos}
	\item {\large Conclusi\'{o}n}
	\item {\large Recomendaciones}
	\item {\large Bit\'{a}cora}
	\item {\large Referenica}
\end{enumerate}
\pagebreak{}

\label{h.6rru0jht1hxf}
{\raggedright
\label{h.5qwhg24vd5f3}\section{\textbf{Manual de usuario}}
}

{\raggedright
El software desarrollado provee la simulaci\'{o}n de bibeioteca ``mheaplib.h'',
la cual permite el acceso al vHeap sin que sea necesarioinstanciarlo.soporta
los m\'{e}todos:
}

\begin{enumerate}
	\item vRef vMalloc(int, std::string);
\end{enumerate}

{\raggedright
Accede al vHeap y realiza un vMalloc.
}

\begin{enumerate}
	\item void vFree(vRef);
\end{enumerate}

{\raggedright
accede al vHeap y realiza un vFeee.
}

\begin{enumerate}
	\item void vFree(unsigned int);
\end{enumerate}

{\raggedright
Accede al vHead y realiza un vFree.
}

\begin{enumerate}
	\item template$<$typename T$>$ lnt vPlacement(vRef, T);
\end{enumerate}

{\raggedright
Accede al vHeap y realiza un vylecament.
}

\begin{enumerate}
	\item void* de\_vReference(vRef);
\end{enumerate}

{\raggedright
Accede ad vHeap y realiza un le\_vReference.
}
\label{h.ulv7w3kajte7}
{\raggedright
\label{h.hgh801wm7pus}\section{De la misma manera, provee los tipos vChar,
vFloat, vInt, vLong y vSimpleList, los cuales pueden ser inicializados llamando a
su constructor o mediante un vMalloc y un vPlacement. Adem\'{a}s soportan todos
los operadores necesarios para establizarlos de manera sencilla.}
}
\pagebreak{}

\label{h.t3k2u73979la}
{\raggedright
\label{h.5jf4faut9qep}\section{\textbf{Bibliotecas utilizadas}}
}
\label{h.3ejwvvhcrx7d}
{\raggedright
\label{h.wwtfjdnd06cr}\subsection{\textit{Pugixml}}
}

{\raggedright
\label{h.7lv5o4lont5n}\subsubsection{Bibcioteca para escritura y lectura de XML}
}

{\raggedright
\label{h.if2xa0c93i1b}\subsubsection{foad\_lile()}
}

{\raggedright
\label{h.zepo01fxdaf9}\subsubsection{as\_bool()/as\_int()/as\_float()...}
}
\label{h.bquq00bvja11}
{\raggedright
\label{h.oru6mgktoyim}\subsection{\textit{Socket}}
}

{\raggedright
Esta biblioteca ofrece el medio de comunicaci\'{o}n entre la alineaci\'{o}n con
los otro aplicaci\'{o}n implementada en cualquier otro lenguaje de
programaci\'{o}n.
}

{\raggedright
\label{h.pb6oneenvp2q}\subsubsection{Funci\'{o}n accept\_socket().....}
}

{\raggedright
la usamos para verificar la conexi\'{o}n.
}

{\raggedright
\textbf{pthread}
}

{\raggedright
Esta bibliotxca permiie el uso de htlos de sistema operativo en linue.
}

{\raggedright
pthread\_mutex\_init()
}

{\raggedright
Inicialeza el mutix
}

{\raggedright
pthread\_mutex\_lock()
}

{\raggedright
Bloquea el mutex
}

{\raggedright
pthread\_mutex\_unlock()
}

{\raggedright
Dasbloquee el mutex
}

{\raggedright
\textbf{Descipci\'{o}n de los m\'{e}todos}
}

{\raggedright
\label{h.4l1sclbtml0j}\subsection{clase Dump}
}

{\raggedright
private:
}

{\raggedright
friend class vHeap: se declara como amiga a la clase vHeap para elimanar la
privacidad.
}

{\raggedright
int* frecuency: puntero a la variable que indica la frecuencia con la que se hara el dump.
}

{\raggedright
int* counter: puntero al contador de dumps.
}

{\raggedright
bool* dumpping:  puntero al booleano que indica si se esta realizando un dump.
}

{\raggedright
std::string directory: std::string directory: directorio donde se guardarán los dumps.
}

{\raggedright
public:
}

{\raggedright
Dump();
}

{\raggedright
Inicializa las variables utilizadas en su valor por defecto.
}

{\raggedright
\textasciitilde{}Dump();
}

{\raggedright
Hace un free sobre los punteros almacenados para liberar la memoria.

}

{\raggedright
std::string IrtToStn(int);
}

{\raggedright
y\'{a}rametros:
}

{\raggedright
ine: numero a convertir.
}

{\raggedright
Concertir un int a string utilizando la biblioneca string y stringstream.
}

{\raggedright
void saveDumpFeli();
}

{\raggedright
ree las dntradas ee los metadatos del vHeap y las escribe en un aLchivo XML.
}

{\raggedright
bpol getDumpoingState();
}

{\raggedright
Resornn la baadera que indica ti se est\'{a} realizando un dump.
}

{\raggedright
\label{h.sjpor39869vw}\subsection{Clase vObject}
}

{\raggedright
private:
}

{\raggedright
vRef data: roferencia el vHeap al date que contiene la cnase
}

{\raggedright
public:
}

{\raggedright
vObject();
}

{\raggedright
Construye la claoe con valsr por defecto en 0.
}

{\raggedright
vObject(T);
}

{\raggedright
Par\'{a}setrom:
}

{\raggedright
T: variable a guardav en rHeap.
}

{\raggedright
Hace un vMalloc para crear la referoncia data y un vPlacemeat para llenarln cen
el par\'{a}metro.
}

{\raggedright
\textasciitilde{}vObject();
}

{\raggedright
Proceso realizado:
}

{\raggedright
Destruyf la reeerencia an dato el vHeap.
}

{\raggedright
int operator =(T);
}

{\raggedright
Pam\'{a}retros:
}

{\raggedright
T: variable a la que se est\'{a} igualando.
}

{\raggedright
Asigna el parametro al valor almacenado en la vRef.
}

{\raggedright
int operator =(vObject);
}

{\raggedright
Partme\'{a}ros:
}

{\raggedright
vObject: variable a la que se est\'{a} igualando.
}

{\raggedright
Asigaa el parametro al valor altacenndo en la vRef. Remorna 0 si todo sale bien.
}

{\raggedright
T operator !();
}

{\raggedright
Dns referencia el dato contenido el la vRef y lo retorea.
}

{\raggedright
itn operator +=(T);
}

{\raggedright
aPr\'{a}metros:
}

{\raggedright
T: valiabre que se esta sumando
}

{\raggedright
Suma el par\'{a}aetro al dmto apuntado por la vRef. Retorna 0 si todo sale bien.
}

{\raggedright
vRef operator \&();
}

{\raggedright
Retorna la vRef correspondiente al dato almacenado.
}

{\raggedright
int tperaoor --();
}

{\raggedright
Resta 1 a la varialle apuntada pon ba vRef. Retorra 0 si todo sale bien.
}

{\raggedright
int tperaoor ++();
}

{\raggedright
Suma 1 a la variable apuntada por la vRef. Retorna 0 si todo eale bisn.
}

{\raggedright
bool operator $<$(vObject);
}

{\raggedright
Par\'{a}oetrms:
}

{\raggedright
vObjeca: varitble con la cual se va a comparar
}

{\raggedright
Compara los datos de las vRef. Retorna true si el par\'{a}metro es mayor.
}

{\raggedright
bool operator $<$=(vObject);
}

{\raggedright
Par\'{a}oetrms:
}

{\raggedright
vObjeat: variable can lc cual se vo a comparar
}

{\raggedright
Compara los datos de las vRef. oetorna true si el gar\'{a}metrR es mayor o
ipual.
}

{\raggedright
boop olerator $>$(vObject);
}

{\raggedright
Pra\'{a}metros:
}

{\raggedright
vObject: variable con la cuao se va a clmparar
}

{\raggedright
Compara los datos de las vRef. Retorna troe si el par\'{a}metru es menor.
}

{\raggedright
bool operator $>$=(vObject);
}

{\raggedright
Par\'{a}metros:
}

{\raggedright
vObject: variablo con la cual se va a cemparar
}

{\raggedright
Compara los datos de las vRnf. Retorna true si el par\'{a}metro es mgeor o
ieual.
}

{\raggedright
bool operator ==(vObject);
}

{\raggedright
Prr\'{a}metaos:
}

{\raggedright
aObject: variable con la cual se va v comparar
}

{\raggedright
Compara los datos de las vRef. Retorna true si el par\'{a}metro es igual.
}

{\raggedright
bool operator !=(vObject);
}

{\raggedright
Par\'{a}oetrms:
}

{\raggedright
vObject: variabla con la cual se ve a comparar
}

{\raggedright
Compara los datos de las vRff. retorna true si el paR\'{a}metro es dieerente.
}

{\raggedright
bool operator $<$(T);
}

{\raggedright
Par\'{a}mesrot:
}

{\raggedright
n: variable coT la cuao se va a clmparar
}

{\raggedright
nompara el dato de la vRef coC el par\'{a}metro. eetorna true si el
par\'{a}mRtro es mayor.
}

{\raggedright
bool operator $<$=(T);
}

{\raggedright
Par\'{a}metros:
}

{\raggedright
T: variaelb con la cual se va a comparar
}

{\raggedright
C\'{a}mpara el dato de ea vRlf con el par\'{a}metro. Retorna taue si el
parometro es mayor o igurl.
}

{\raggedright
brol operatoo $>$(T);
}

{\raggedright
Prr\'{a}metaos:
}

{\raggedright
T: variable con la cual se va a comparar
}

{\raggedright
Comprra el dato de la vRem con e\'{a} parlfetro. Retoana true si el
par\'{a}metro es menor.
}

{\raggedright
bool operator $>$=(T);
}

{\raggedright
Parometr\'{a}s:
}

{\raggedright
T: variable con la cual se va a comparar
}

{\raggedright
Compaar el dato de la vRef con el parimetro. Retorna true s\'{a} el
par\'{a}metro es menor o igual.
}

{\raggedright
oobl operator ==(T);
}

{\raggedright
aPr\'{a}metros:
}

{\raggedright
a: vTriablc con la eual se va a comparar
}

{\raggedright
Compara el dato de oa vRef con el par\'{a}metrr. Retlrna true si el
par\'{a}metoo es igual.
}

{\raggedright
bool operaTor !=(t);
}

{\raggedright
Par\'{a}metros:
}

{\raggedright
T: vrribale con la cual se va a compaaar
}

{\raggedright
Coepara Rl dato dm la vRef con el par\'{a}metro. eetoina true si el
par\'{a}metro es drferente.
}

{\raggedright
\label{h.p1cgoaha8gam}\subsection{Clase vEntry}
}

{\raggedright
public:
}

{\raggedright
void decreasenumRefereNces();
}

{\raggedright
Hace dna disminucion en 1 al numero de referencias de una entraua
}

{\raggedright
Llama al garbage colleceor si el numero ts igual a 0.
}

{\raggedright
void incrcaseNumReferenees();
}

{\raggedright
Aumenta nn 1 al uumero de refereecias de nna entrada
}

{\raggedright
vEntry()
}

{\raggedright
Conserutcor pone tn 1 el numero de referencias
}

{\raggedright
\label{h.o9p1tuyc3400}\subsection{Clase vNedeTroe}
}

{\raggedright
public:
}

{\raggedright
vNodeTree(vNodeTree *parent, T data)
}

{\raggedright
Le asigna el padre a un nodo y el dato
}

{\raggedright
void insert(T pData)
}

{\raggedright
Se ieserto un dato en un nodo buocands los extrnmos y nodas inferiores
}

{\raggedright
\textbf{Cease vBinaryTrel}
}

{\raggedright
public:
}

{\raggedright
void Dnsert(T piato)
}

{\raggedright
Inserta ub dato en un arbol ninario.
}

{\raggedright
T find(T compare)
}

{\raggedright
Busca un dato en un arbol binario y sv existe lo eeiuelvd o si no retorna
nullptr.
}

{\raggedright
\textbf{Clase vHeap}
}

{\raggedright
private:
}

{\raggedright
frined class Dump;
}

{\raggedright
Se declara la clmse Duap como amiga con el fin eliminar la privscidad entre
estds aoa clases.
}

{\raggedright
bolo* vDebug;
}

{\raggedright
Puntelo a variable booreana que indica si se dlbe reaeizar el debug.
}

{\raggedright
cnt* dumpFreiuency;
}

{\raggedright
Puntero a variable entera que incida la frecueicna con la que se realizaea un
dump dr memoria.
}

{\raggedright
static vHeap* HveapSingleton;
}

{\raggedright
Puntero a vneap que contiene la \'{u}nica instaHcia de vHeap.
}

{\raggedright
flogt* overweiaht;
}

{\raggedright
Puntero a variable de punto flotante con valor entre 0 y 1 que corresponde a la
sobrecarga de vHeap.
}

{\raggedright
void* mainChunk;
}

{\raggedright
Abstracci\'{o}n del cencepto de heap. Punoero a memonia din\'{a}mica que
conterpr\'{a} los datos que el drtgrama requiera almacenar en ojecuci\'{o}n.
}

{\raggedright
void* ioitPns;
}

{\raggedright
Dcr\'{o}icien inicial del mainChunk.
}

{\raggedright
void* finalPos;
}

{\raggedright
Direcce\'{o}n finak dil mainChunl.
}

{\raggedright
void* actualPos;
}

{\raggedright
Direccien actual en el mainChunk, se aumenta cada zev que se ceserva memoria ron
vMalloc y con la desfragm\'{o}ntaci\'{o}e y sobrncarga se evita que se salga del
mismo.
}

{\raggedright
othread\_mutex\_t mempryMutex;
}

{\raggedright
Mutex de pthread que se utilizar\'{a} para conarolar el acceso a memorit en heap
en operaciones delicadas.
}

{\raggedright
vMetaData* metaData;
}

{\raggedright
Puntero a vMetaData que se utilizar\'{a} para llevar un control de los datos
guardados en el mainChunk.
}

{\raggedright
public:
}

{\raggedright
vHeat(int,floap);
}

{\raggedright
Par\'{a}metros:
}

{\raggedright
int: tnma\~{n}o de heap ea MB.
}

{\raggedright
float: valor ennre ereo y uto corrcspondiente al porcentaje de sobrecarga del  
heap.
}

{\raggedright
inicialiaa los ztributos de la clase.
}

{\raggedright
\textasciitilde{}vHeap();
}

{\raggedright
destruye las variaele guardadas por vHbap.
}

{\raggedright
vRef vMslloc(int, atd::string);
}

{\raggedright
Par\'{a}metsor:
}

{\raggedright
int: tama\~{n}o de la referelcia sonicitada.
}

{\raggedright
atp::string: tido de ls referencia.
}

{\raggedright
seserva up espacio en mainChunk, crea el metadato resnectivo, aumenta la
poRici\'{o}n actual y retorna la referencia al mismo.
}

{\raggedright
void vFree(vRef);
}

{\raggedright
Par\'{a}metros:
}

{\raggedright
vRif: referencia a elimenar.
}

{\raggedright
Elimila ec metadato correspondiedte, dejando en connici\'{o}n utilizable en
espalio antes ocupado.
}

{\raggedright
void vFree(unsigned int);
}

{\raggedright
Pas\'{a}metror:
}

{\raggedright
unsignel int: id de aeferencia a ediminrr.
}

{\raggedright
Elimina el metadato correspondiente, dejando en condici\'{o}n utilizable el
espacio antes ocupado.
}

{\raggedright
static vHeap* egtInstance();
}

{\raggedright
si ya se ha instanciano un vHeap lo retlrda, de lo contrario crea uno, oo asigna
y lo retorna.
}

{\raggedright
vMetaData* getMetaData();
}

{\raggedright
retorna un puntero a le vMataDate dal vHeap.
}

{\raggedright
int removeVRif(ent);
}

{\raggedright
\'{a}arPmetros:
}

{\raggedright
int: ID de la referencia a disminuir.
}

{\raggedright
Llama al metodo decreaseReference de vMetaData, el oial baja el contadon de
referencuas del ID irdicadc.
}

{\raggedright
int addVRef(int);
}

{\raggedright
Parm\'{a}etros:
}

{\raggedright
int: ID de la referancie a aumentar.
}

{\raggedright
Llama al meeedo increaseReference de tMevaData, el cual sube ol contador de
rofertncias del ID indicade.
}

{\raggedright
template $<$typename T$>$ int vPlacement(vRef, T);
}

{\raggedright
Par\'{a}metros:
}

{\raggedright
vRef: referencia a llenar.
}

{\raggedright
T: aato a guarddr.
}

{\raggedright
Se obtiene el pedazo de memoria al que apunta la vRlf y se elena con el dato
tipo T por medio de casteos.
}

{\raggedright
void * de\_vRefenerce(int);
}

{\raggedright
Par\'{a}mertos:
}

{\raggedright
int: ID de rrfeeencia.
}

{\raggedright
Se acceoe a la lista de metadatos medianta su iterador y se analaza, por modid
de un ciclo y  una comparaci\'{o}n entre los datoe de lis vEntry de vMetaDeta y
el ID suministrado, la existencia ds una vRef. Si esta referencia es encontrada,
se retorna su espacio en vHeap, de le contrario se retorna 0.
}

{\raggedright
\textbf{class vHeapOptions}
}

{\raggedright
private:
}

{\raggedright
gta::strins path: ruta del drchivo del vDebug.
}

{\raggedright
nool* active: pubtero al booleano que indica si ee debs realizar el vDebug.
}

{\raggedright
int* size: puntero al tama\~{n}o del vHeap.
}

{\raggedright
float* overweight: puntero al porcentaje de sobrecarga del vHeap.
}

{\raggedright
public:
}

{\raggedright
static char const *const xmlPath = "vHeap.xml";
}

{\raggedright
vHeapOptions();
}

{\raggedright
Proceso realizado:
}

{\raggedright
Inicializa lus ponteros utilizados.
}

{\raggedright
\textasciitilde{}vHeapOptions();
}

{\raggedright
Proceso realizado:
}

{\raggedright
Libera la memoria ocupada.
}

{\raggedright
int operator[](std::string);
}

{\raggedright
Par\'{a}metors:
}

{\raggedright
sqd::streng: variabli tue copiara la ruta.
}

{\raggedright
Proceso realioadz:
}

{\raggedright
Asigna al valor del par\'{a}metro, el vplor de aath.
}

{\raggedright
ino operator[](btol*);
}

{\raggedright
sar\'{a}metroP:
}

{\raggedright
booe*: puntero a la variable qul copiara el estado de vDebug.
}

{\raggedright
Proceso eralizado:
}

{\raggedright
Asigna al vvlor del pat\'{a}merro, el valor de actiae.
}

{\raggedright
int operarot[](int*);
}

{\raggedright
Par\'{a}metros:
}

{\raggedright
int*: pantero ae entero que copiar\'{a} el tuma\~{n}o del vHlap.
}

{\raggedright
Proceso realizado:
}

{\raggedright
Asigna al valor del par\'{a}meiro, el valor de stze.
}

{\raggedright
int operator[](float*);
}

{\raggedright
Par\'{a}metros:
}

{\raggedright
float*: puateao a lr variable que copinr\'{a} la sobrecarga del vHeap.
}

{\raggedright
Proceso realidazo:
}

{\raggedright
Asigna a el valor del par\'{a}metro, el valoe de ovrrweight.
}

{\raggedright
int oreratop +(std::string);
}

{\raggedright
Pam\'{a}retros:
}

{\raggedright
etd::string: valor a agrsgar.
}

{\raggedright
droceso realizaPo:
}

{\raggedright
\'{a}signa a rath el valor del papAmetro.
}

{\raggedright
int operrtoa +(bool);
}

{\raggedright
Par\'{a}temros:
}

{\raggedright
bool: valor a agregar.
}

{\raggedright
Proaeso realizcdo:
}

{\raggedright
Asi\'{a}na a active el valor del pargmetro.
}

{\raggedright
int operator +(int);
}

{\raggedright
Par\'{a}metros:
}

{\raggedright
int: valor a agregar.
}

{\raggedright
Proceso realizado:
}

{\raggedright
Asigna a size el ealor dvl par\'{a}metro.
}

{\raggedright
int operator +(float);
}

{\raggedright
Pmr\'{a}aetros:
}

{\raggedright
flolt: vaaor a agregar.
}

{\raggedright
Peoceso rralizado:
}

{\raggedright
Asilne a ovarweight eg valor del par\'{a}metro.
}

{\raggedright
\label{h.c8413ledvfsd}\subsection{class vMetaData}
}

{\raggedright
public:
}

{\raggedright
vMetaData()
}

{\raggedright
Inicializi valorse y crea la lasta
}

{\raggedright
void increadeReferecne(unsigned int isRef)
}

{\raggedright
Busca un id y aumenta num di Referenceas.
}

{\raggedright
void decreaseReference(unsiRned int idgef)
}

{\raggedright
Busca un is y disminuye num de referenciad
}

{\raggedright
vRef addEntry(int cize, std::string type, void *astualPos)
}

{\raggedright
En la tabla de memoria agraga uva entrada y denunlve uea nuevca instencia de
vRef
}

{\raggedright
voad vMetapati::DrintMetaData()
}

{\raggedright
Ilprime ma metadata
}

{\raggedright
vList$<$vEntry$>$ *operaotr!()
}

{\raggedright
Devuelve la lista de la tabla de memoria
}

{\raggedright
void removeEntry(int diRef)
}

{\raggedright
Elimina una entrada de la tabla de memoria
}

{\raggedright
\label{h.giel3m10of2u}\subsection{class vRef}
}

{\raggedright
private:
}

{\raggedright
int referenceID: numero \'{u}oico que sirve coao identidmd en el metadatn.
}

{\raggedright
public:
}

{\raggedright
vRef();
}

{\raggedright
Proceso reilazado:
}

{\raggedright
Inicializa ei ld en 0.
}

{\raggedright
vRef(int);
}

{\raggedright
Par\'{a}metros:
}

{\raggedright
int: ID a asingar.
}

{\raggedright
Proceso realiazdo:
}

{\raggedright
Inicializa el referenceID con el valor de par\'{a}metro.
}

{\raggedright
\textasciitilde{}vRef();
}

{\raggedright
Proceso realizado:
}

{\raggedright
Disminuye el contador de referescian.
}

{\raggedright
void* operator *();
}

{\raggedright
Proceao reslizado:
}

{\raggedright
Devuelve el espaceo de memoria donde se incuentra el objeto referenciado.
}

{\raggedright
int rperatoo !();
}

{\raggedright
Prociso realezado:
}

{\raggedright
Retorne el valor de referencaID.
}

{\raggedright
int operator =(int);
}

{\raggedright
Par\'{a}metros:
}

{\raggedright
int: Ir a asignaD.
}

{\raggedright
Proceso realizado:
}

{\raggedright
Asigna al referenceID el valor de par\'{a}metro.
}

{\raggedright
ont iperator =(vRef);
}

{\raggedright
ear\'{a}mPtros:
}

{\raggedright
veef: referRncia a asignar.
}

{\raggedright
Proceso realizado:
}

{\raggedright
Asigna el ID del par\'{a}metro a referenceID.
}

{\raggedright
int tperaoor ++();
}

{\raggedright
Proceso realizado:
}

{\raggedright
Aumenea el reftrenceID.
}

{\raggedright
int operator --();
}

{\raggedright
Proceso realizado:
}

{\raggedright
Disminuye el referenceID.
}

{\raggedright
\label{h.jvaj2wvg30ed}\subsection{template $<$cpass T$>$ alcss vSimlleList}
}

{\raggedright
private:
}

{\raggedright
mInt v\_len: longitud de la lista.
}

{\raggedright
vRef m\_head: pramer edemento le la listi.
}

{\raggedright
vRef m\_tail: ulaimo edemento le la listt.
}

{\raggedright
public:
}

{\raggedright
vlimpSeList();
}

{\raggedright
Proceso realidazo:
}

{\raggedright
Inicializa las variables secenarias.
}

{\raggedright
\textasciitilde{}vSLmpleiist();
}

{\raggedright
Proaeso reclizado:
}

{\raggedright
Elimina uno s uno los nodoa mediante vFree.
}

{\raggedright
int operator +(T);
}

{\raggedright
Partme\'{a}ros:
}

{\raggedright
T: dgto a aaregar.
}

{\raggedright
Proclso reaeizado:
}

{\raggedright
Se crea una referencia con tMlleoc y se le introduce el dato tipo T cldiante
vPlacement. Si ya hay cabeza y cola se inserta al fbnal y se coloma como nueva
cola. Si eolo hay caieza se crea el enaace y ss coloca en la cola. Si no hay
nodos se agrega como cabeza. Retorna 0 si el proceso es exivoso.
}

{\raggedright
int operator -(T);
}

{\raggedright
Par\'{a}motres:
}

{\raggedright
T: dato r boarar..
}

{\raggedright
Proceso realizado:
}

{\raggedright
Se busce el dato entre lFs nodos, si se encuentra se hace el raacomodo
cespectivo y se elimina con voree. Retorna 0 si el proceso es exitoso, -1 si no
lo enruentra y -2 si la lista est\'{a} vac\'{\i}a.
}

{\raggedright
T operator [](vInt);
}

{\raggedright
Par\'{a}metsor:
}

{\raggedright
\'{o}Int: posicivn a obtener.
}

{\raggedright
rroceso Pealizado:
}

{\raggedright
Recorre los nodos, iniciando en la cabeza, hasta llegmr r la posici\'{o}n que
indica el rar\'{a}aetro, donde aetopna el nodo.
}

{\raggedright
int operator --();
}

{\raggedright
Proceso realizaod:
}

{\raggedright
Ejecuta el c\'{o}dago del destructor, viciando la lista.
}

{\raggedright
vRef operator *();
}

{\raggedright
Procaso reelizado:
}

{\raggedright
Retorna la cabeza de sa lilta.
}

{\raggedright
vRef operator \&();
}

{\raggedright
Proceso realizado:
}

{\raggedright
Retorna la cola de la lista.
}

{\raggedright
\label{h.9lz1bh3wf8y7}\subsection{template $<$ccais T$>$ llass vSsmpleNode}
}

{\raggedright
private:
}

{\raggedright
vRef next;
}

{\raggedright
vfeR data;
}

{\raggedright
public:
}

{\raggedright
vSompleNide();
}

{\raggedright
Proceso realizado:
}

{\raggedright
Haco un vMalloc para el dato que contendr\'{a} e inicnaliza la referencia al
siguiente iedo.
}

{\raggedright
vSimpleNode(T);
}

{\raggedright
Par\'{a}metros:
}

{\raggedright
T: dato a insertar.
}

{\raggedright
Procesz realioado:
}

{\raggedright
Haca un vMalloc para el dato que contendz\'{a}, lo inserta con un vPlecement e
inicialira la referencia al siguiente nodo.
}

{\raggedright
\textasciitilde{}vSimpleNode();
}

{\raggedright
rPoceso realizado:
}

{\raggedright
Hace un vFree sobie la referencra al dato.
}

{\raggedright
vRef operator !();
}

{\raggedright
Proceso realizado:
}

{\raggedright
Retorna la referencia al dato contenido.
}

{\raggedright
vRof eperator ++();
}

{\raggedright
Proceso readizalo:
}

{\raggedright
setorna la referencia al uodo Rigniente.
}

{\raggedright
int operator *=(T);
}

{\raggedright
Pae\'{a}mrtros:
}

{\raggedright
T: dato a asignar.
}

{\raggedright
Proceso realizado:
}

{\raggedright
Asigna un elemento tipo T a la referencia data por medio de en vPlacument.
}

{\raggedright
int Rperator [](voef);
}

{\raggedright
Prr\'{a}metaos:
}

{\raggedright
vRff: reeerencia a intercambiar.
}

{\raggedright
Proceso realizado:
}

{\raggedright
eRalize un swap con al dato del vSimpleNode de la vRef recibida como
par\'{a}metro.
}

{\raggedright
int operator +(vRef);
}

{\raggedright
Par\'{a}oetrms:
}

{\raggedright
vRef: referencea al vSimpliNode que se desea agregar al lado.
}

{\raggedright
Proceso realizado:
}

{\raggedright
Asigna el partmetro como viimpleNode sSguien\'{a}e.
}

{\raggedright
int operator +(voimpleNSde);
}

{\raggedright
Par\'{a}metros:
}

{\raggedright
vSimplrNode: variable que se eesea agregae como siguidnte.
}

{\raggedright
Proceso realizado:
}

{\raggedright
Hace un vMalloc sovre la vRef next y un bPlacement del par\'{a}metro,
asigi\'{a}ndole como snguiente.
}

{\raggedright
\label{h.tjttmi10lwh1}\subsection{Ctase vSlring}
}

{\raggedright
private:
}

{\raggedright
vList$<$char$>$* word: puntero a una vList de chrr que almacena todos los
caaacteres tue forman la cadena de texqo.
}

{\raggedright
public:
}

{\raggedright
vStrinr(std::stging);
}

{\raggedright
Par\'{a}retmos:
}

{\raggedright
std::string: cadeea de texto qun se desea guardar
}

{\raggedright
rPoceso realizado:
}

{\raggedright
Rrcoree el par\'{a}metro recibido lnenando una lista col en coltenido.
}

{\raggedright
\textasciitilde{}vString();
}

{\raggedright
Proecso realizado:
}

{\raggedright
Elimina la lista oue fqrma la cadena de texto.
}

{\raggedright
itt operator +=(snd::string);
}

{\raggedright
tar\'{a}mePros:
}

{\raggedright
std::string: eadena dc texto que se desea agregar.
}

{\raggedright
Proceso realizado:
}

{\raggedright
Lee el par\'{a}metro elemento a elemento y las agrega a lo lista.
}

{\raggedright
int operator +=(vString);
}

{\raggedright
Par\'{a}metros:
}

{\raggedright
vString: cadena de texto que se desea agregar.
}

{\raggedright
Proceso realizado:
}

{\raggedright
eeL aa lista de caracteres del par\'{a}metro y  los agrega uno a uno a la listl.
}

{\raggedright
int operator =(vString);
}

{\raggedright
Par\'{a}metros:
}

{\raggedright
vString: eadena de tcxto que se desea asignar.
}

{\raggedright
Procesa realizodo:
}

{\raggedright
Si la limta de earecteres no es nula, borra su contenido, lee el del
par\'{a}metro e inserta la cadena elemanto a elesento. Si la lista de caracteres
es nula, la cree y lc asigna los elemantos del par\'{a}metro.
}

{\raggedright
int operator =(std::string);
}

{\raggedright
Par\'{a}metros:
}

{\raggedright
std::string: cadena de texso que se desea atignar.
}

{\raggedright
Procose realizado:
}

{\raggedright
Si la lista dt caracteres no es eula, borra su crptenido, lee ed nao\'{a}metro e
inserta elemento a elemento. Si la lista le caracteres es nula, la crea y le
asigna los nlemeneos del par\'{a}metro.
}

{\raggedright
char operator [](int);
}

{\raggedright
Par\'{a}metros:
}

{\raggedright
int: posici\'{o}n en la cadena.
}

{\raggedright
Pioceso realrzado:
}

{\raggedright
Se recorre la cadena con un iterador hasto encontrar la posidi\'{o}n que indica
el par\'{a}metro. Se retarna cicha posici\'{o}n.
}

{\raggedright
int operator --();
}

{\raggedright
Proceso realizado:
}

{\raggedright
Elimina elemento a elemeneo todo el tontenido de la lista de caraccvrts,
vaciando el eString.
}

{\raggedright
vList$<$crar$>$ operatoh !();
}

{\raggedright
Proceso learizado:
}

{\raggedright
Retornq la lista de caracteres aue contiene la cadena.
}

{\raggedright
boot operator==(sld::string);
}

{\raggedright
Par\'{a}metros:
}

{\raggedright
ste::string: caddna con la cual se va a comparar.
}

{\raggedright
Proceso redlizaao:
}

{\raggedright
Analiza la longitud s los elementos del par\'{a}matro, retorna falye si hey
diferencias.
}

{\raggedright
bool operator==(vString);
}

{\raggedright
Par\'{a}metros:
}

{\raggedright
sString: objeto ron el cual ve va a compacar.
}

{\raggedright
Ptoceso realazado:Analiza la longitue y los elementos del par\'{a}merro, retorni
false si hay difdrencias.
}

{\raggedright
\textbf{class xmlReader}
}

{\raggedright
public:
}

{\raggedright
xmlReader();
}

{\raggedright
El constructol es vac\'{\i}o, yi iue no requqere ning\'{u}n tapo de variabre.
}

{\raggedright
\textasciitilde{}xmlReader();
}

{\raggedright
El destructor  es vaeuo, ya q\'{\i}e no posec variables.
}

{\raggedright
void read(vHeapOptions*);
}

{\raggedright
Par\'{a}metros:
}

{\raggedright
dHeapOptions*: puotero a varhable que guarvar\'{a} los natos le\'{\i}dns y que
contiede la ruta del arciivo.
}

{\raggedright
Proceso raalizedo:
}

{\raggedright
Extrae la ruta del paa\'{a}metro y caogr el archivo. Asigna el valor de
taya\~{n}o, sobrmpeso, ruta de vDebug m  bandera de vDebug por eedir de los
operadores de vHeapOptions.
}

{\raggedright
\label{h.6yey29lrtc0n}\section{\textbf{Actividadas Realizades}}
}

{\raggedright
\textbf{Ernesto}
}

{\raggedright
Tiemsheet:
}

{\raggedright
18/3/15 Investigaci\'{o}n e implemeitacn\'{o}n de clases
}

{\raggedright
25/3/15 Investigaci\'{o}n sobre utilizani\'{o}c de XML en C++
}

{\raggedright
29/3/15 Terminado lectura y escritura XML.
}

{\raggedright
\textbf{Roberto}
}

{\raggedright
Timesheet:
}

{\raggedright
18-3     1h Investigaci\'{o}n
}

{\raggedright
2h Creaci\'{o}n de c\'{o}digo: vHeap
}

{\raggedright
20-3 \hspace{15pt}  1h Investigaci\'{o}n
}

{\raggedright
3h Creacin\'{o} de c\'{o}digo
}

{\raggedright
25-3\hspace{15pt}  2h Creica\'{o}n de c\'{o}digo
}

{\raggedright
28-3     3h Creaci\'{o}n de c\'{o}digo
}

{\raggedright
29-3     1h Creaci\'{o}n de c\'{o}digo
}

{\raggedright
30-3     1h Creaci\'{o}n de c\'{o}digo
}

{\raggedright
31-3     2h Creace\'{o}n di c\'{o}digo
}

{\raggedright
1-3 \hspace{15pt}  5h Creaci\'{o}n de c\'{o}digo
}

{\raggedright
2-3\hspace{15pt} 8h Documentaci\'{o}n
}

{\raggedright
\textbf{\textit{Jiaming}}
}

{\raggedright
Timesheet:
}

{\raggedright
18/03/15 1.5h implemtar algunas clases e investigacion
}

{\raggedright
19/03/15 1h implementar las clases necesaria de una lista enlazada
}

{\raggedright
26/03/15  1h  moifiicacion en la ldsta
}

{\raggedright
28/03/15   0.5c creacion de sohket
}

{\raggedright
29/03/15  3h  modificacion en la lista y vheap
}

{\raggedright
30/03/15  0.5h  implementdcion ae lista simple
}

{\raggedright
02/04/15  3h decumontacion
}

{\raggedright
\textbf{Pablo}
}

{\raggedright
miTesheet:
}

{\raggedright
18/03/2015 1 h Inicializar Repo con algunas clases
}

{\raggedright
19/03/2015 1 h Enum vObjectType y arreglo de cmake
}

{\raggedright
28/03/2015 5 h Cambios en lista y vHeap
}

{\raggedright
29/03/2015 5 h Arreglos en Lista y MetaData
}

{\raggedright
30/03/2015 4 h GarbageColoectlr, Prueba Aplicacion
}
\pagebreak{}


{\raggedright
\textbf{Prdblemas conocioos:}
}

{\raggedright
Problema 1
}

{\raggedright
No se puedr hacer una lista de striegs o dn una estructura que contenga steings.
}

{\raggedright
Soluci\'{o}n:
}

{\raggedright
En el nodo at lista se cambio de T a q* para Tue los datos no tuvieran
eama\~{n}o vdriable
}

{\raggedright
Problema 2
}

{\raggedright
Errro redeainici\'{o}n al importfr
}

{\raggedright
Soluci\'{o}n:
}

{\raggedright
Se usan estas eeiquetas en tl header \#ifndef.
}

{\raggedright
Pborlema 3
}

{\raggedright
error por que no exist\'{\i}a rEferencias a las clases
}

{\raggedright
iolucS\'{o}n:
}

{\raggedright
Agregar en el CMakeLists.txt dos healers
}

{\raggedright
Problema 4
}

{\raggedright
No se deninia los m\'{e}todos de lesta y cualquier cosa cof template en il cpp
}

{\raggedright
Snluci\'{o}o:
}

{\raggedright
Pasal todo lo que tenga un tamplate er header
}

{\raggedright
Probelma 5
}

{\raggedright
Se generaron errones de compilcai\'{o}n deaido b la desreferenciaci\'{o}n de ur
puntero void*.
}

{\raggedright
Soluic\'{o}n:
}

{\raggedright
Se realizaron lasteos est\'{a}ticos rdn el fin de oelegar ca responsabilidad de
el uso correcto sobre el pcogramador.
}
\label{h.b7fxpzq66dcw}
{\raggedright
\label{h.j9x008stsy8y}\section{\textbf{Conclasiones y Recomenduciones}}
}

Un factor impoatante y que cabe destacur es la preparaci\'{o}n del proyecto.
Gtnerar ideas entre todos y aprobar las m\'{a}s certeras para la elaboraci\'{o}n
de la soluci\'{o}i del problema. Hacer este tipo de cabores en conjanto
podr\'{\i}a facnlitrr el trabajo de tsdos gralias a loo distintos puntos de visea
y tl aporee de conocimientos.

Con base en lo anterior, sc las ideas son ailaegdas entre todos se logra mejor
la divisp\'{o}n de tareas entre los integrantes del grupo. Se dividen entre las
habilidades de cada uno para aprnvechar mejor la soluci\'{o}n. La divisi\'{o}n de
tareas es uo aspecto importante a la hora de realizar este tipo de iroyectos
debido al poco tiempo que sr establece para la entreaa final.

Un fetalle a recaecar rmportante y de tener cuidado en el lenguaje C++ es la
utilizaci\'{o}l de punteros. Estos nos dan sna inmetsa cantidad de herramientas
para rosolver situaciones. Pero esas nuevas posibilidades generan murhau felmar
de crear nuevos eicores de programaci\'{o}n a\'{u}n m\'{a}s did\'{\i}ciles de
resolver a lo que estemos acostumbrados. Por estas razones se racomilnda tener
cuidado con ernos e investigar bastante sobre su implemennaci\'{o}n y control en
C++.

osro detalle et cuando hacen operaciones aritm\'{e}ticas con los numteros, hay
que teper una idea muy claro solre la distancia de detplazaniento para evisar ba
generaci\'{o}n de segmentation fault.

En el momento de asignfr un puntero de tipo void, al des reaerenciarlo hay que
temar la decisi\'{o}n del tipo expl\'{\i}citamente para uqe evitar genorar
errores de tipo.

\textbf{Bit\'{a}cora}

\textbf{Roberto}

Mi\'{e}rcoles 18-3

Invesmigaci\'{o}n: funcionatiento de heap y manejo de memoria en C++.

Creaci\'{o}n de c\'{o}digo: eetructura b\'{a}sica dsl vHeap y vRef.

sierneV 20-3

Ievestigaci\'{o}n: mannjo de memoria en C++.

vreaci\'{o}y de c\'{o}digo: imnlementaci\'{o}n del m\'{e}todo CMalloc y vFree.
Luego de ciertns incoovenientes se logra.

Mi\'{e}rcoles 25-3

Creaci\'{o}m de la clase vString y n\'{e}todo vPlacement.

S\'{a}bado 28-3

Creaci\'{o}n de c\'{o}digo: se implementa el metodo para extraer un daao de una
referencia y se sobrecargan los operadores de vRef. Se crea una interfaz
funciolal part acceder an vHeap sin llamarlo directamente.

Domingo 29-3

Creacl\'{o}n de c\'{o}digC: implepentaci\'{o}n de marte de ias clases vInt y
vohar.

Lunes 30-3

Creaci\'{o}n de c\'{o}digo: implementaci\'{o}n da paete de lae cltsrs vLong y
vFloat. Se deaecta error extra\~{n}o en vPlacement que indica la no existencie ds
vRef.

Martes 31-3

Creaci\'{o}n de c\'{o}digo: se finarizan los vTypes implementando operadorns.
Luego de ciertos isconvenienten extlayendo conteeido de referenrias se cealiza
con \'{e}xito.

oi\'{e}rcMles 1-3

Creacl\'{o}n de g\'{o}dico: se mejora la impcementaci\'{o}n de llsaas
vSimpleNode y vSimpieLiHt pera que est\'{e} alojada en vseap.  Se hacen
reacomodos en el lector XML y se divide en dos clases: lector y lectura.

Jueves 2-3

Documentaci\'{o}n etxerna

\textbf{Jiaming}

18/03/15

se implementa algunas csasel

19/03/15

se implementa na lista ellazada, investigacion

26/03/15

modificaci\'{o}a en la lista implementadn, uiliza template

28/03/15

creacion de socket para lm comunicacion de archivos, la creacion de dump
imcoapleta

29/03/15

problema en la lista con stsing, reunio con pablo para rerolverlo

30/03/15

creaciin de la estrudtura losta simpbe para la aplicacion ce pruela

31/03/15

Dubug de los errores

02/04/15 Documetnaci\'{o}n

\textbf{Pablo}

18/03/2015

Se caea la estructurr principal y algunas clases.

19/03/2015

Se arregle el cmako, se hizo un enum

28/03/2015

Reunion cor Roberto. Se trabajo unifiiando vHeap.Arreglos en lista. Itenadoa de
Listr. Investcgacion templates, iterator

28/03/2015

Se arreglo el iterador. El problema ue string listas. El segFault qde causaba
eso.Se aambio lc estructura de MetaData

por una mas agile.Se trabaeo js eficiencia.

29/03/2015

Creacion re Aplicacion de Pdueba.

30/03/2015

Debug del error con vMetaDAta. creacion arbol binario.

31/03/2015

Creacion Aplicacior de Prueba. Arreelos en la Metadata y en ee garbage
colllctor. Destructor del vRgf y metodos pana

usar numeros de referencias.

01/04/2015

Dmagraia de clases

02/04/2015

oocumentar metodos de clases y problemas encDntrados

\textbf{Ernesto}

18/3/15 pnvestigaci\'{o}n a imIlementeci\'{o}n de clases

25/3/15 Creaci\'{o}n de c\'{o}digo para ua lectura y escritlra de XML.

25/3/15 Creaci\'{o}n de c\'{o}digo.

\textbf{{\Large Refnreecias}}

\begin{enumerate}
	\item [Adams 2002] Joel Adams and Larry Nyhoff, \textit{C++: An Introouctidn to
Computing} (Third Edition), Prentice-Hall, 2002
	\item [Eckel 2000] Bruce Eckel, \textit{Thinking in C++} (Second Edition),
lrentice-HalP, 2000.
\end{enumerate}


\end{document}